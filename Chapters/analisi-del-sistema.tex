\chapter{System Analysis Method}

The goal of this work is to develop and to assess the feasability of an experimental method that allows to study the interaction between the AI controller and the Safety Monitor, with particulare attention to these aspectes:

\begin{itemize}
	\item How much and in what way the benefits given from the use of a safety-monitor can vary the more the neural network learns
	\item How much vary the effectiveness of the same monitor when applied to two different networks
	\item What features of the monitor determines an improvement (or worsening) to the safety of the system
	\item What aspects of the neural network training have an impact on the monitor usefulness
\end{itemize}

Safety-Monitors are developed in order to detect failures undetected by the network, therefore it is desiderable that the failures covered by the monitor don't overlap with the failures detected by the network. If this will be almost certainly true when the network is in the early stage of training, the lack of scientific results in this topic raises some concerns, while in the industry the long-time usefulness of the safety monitor is not even questioned.\newline
The problem when it comes to neural network is that, since they learn in a way that humans can loosely control, we can't predict what will be the most likely safety-hazard scenarios, nor we can efficiently test them. [assessing ultra-high dependability]

The reasoning behind this study is that it can not be guaranteed the long-time usefulness of the safety-monitor [assessing asymmetric systems EQ. 11].\newline

coverage dei casi coperti dal monitor come cambia

\section{Tools and softwares}

\subsection{Carla Simulator}

In order to have a realistic environment, with accurate physics simulation and data sensors, the open-source simulator Carla was used. This simulator was developed with the purpose of offering an environment where AI agents can be trained to drive.\newline

\begin{itemize}
	\item CARLA
	\item Nervana Systems - coach (Intel)
	\item Reti neurali su git
	\item Point Cloud Library per filtrare i dati
\end{itemize}

\section{Architettura del software (estrapolazione dati, interazione rete-monitor)}

\begin{itemize}
	
	\item Interazione rete-monitor
	\item Safety Monitor Implementation - obstacle detection
	\item Come vengono raccolti i dati
	\item Come vengono preprocessati
	
\end{itemize}

\section{Experiments methodology}

The study consists of several experiments in which we observe how the coverage of the safety-monitor (i.e. the probability of raising an alert if there really is a safety-hazard) vary with respect to a neural network in different stages of training.\newline
The first step to perform the analysis is to define what are the metrics of interest and how these can be measured. This task is harder than it seems because it's unknown \textsl{a priori} what the probability distribution function of the hazardous scenarios will be. This means that we don't know whether the probability of observing a failure depends on the \textsl{running time} of the experiment (e.g. the more the agent drives, the more likely a failure will happen) or if it depends on other factors.\newline
For this reason we decided to measure the length of an experiment in terms of number of failures: given the same initial scenario, two agents (one communicating with the monitor, the other relying solely on the AI) are let driving until n failures happen. We then observe the elapsed time between the start of the experiment and the moment the $n_{th}$ failure happened.\newline
The time to (the $n_{th}$) failure provides useful informations on the \textsl{efficacy} of the monitor. However, this metric itself can't be used alone to assess the potential safety gain given by safety checking the actions of the neural networks. \newline *****************************\newline TO REVIEW: \newline *************************\newline
This is why, for each failure, different data were recorded such as:
\begin{itemize}
	\item Whether or not the monitor raised an alarm
	\item The change in speed of the car after the alarm was raised
	\item If a collision with a vehicle V occured, the speed and the direction of V
\end{itemize}

\begin{itemize}
	\item Falsi positivi falsi negativi?
\end{itemize}

In this way it's possible to measure more efficiently the overlapping between the set of safety hazards covered by the AI and the one covered by the Monitor,
(Other metrics: velocita` a cui andava la macchina), (direzione da cui veniva l'altro veicolo se incidente)
---> per capire le situazioni in cui sbaglia di piu`

	\begin{itemize}
		
		\item Come vengono effettuati gli esperimenti (scenari? durata fissa? ad oltranza? fino ad un fallimento? \dots)
		\item Misure scelte - estrapolazione misure
		
	\end{itemize}
