\chapter{System Analysis Method}

In this chapter is presented a method to study the safety level of an autonomous car over time, observing the emergence resulting from the interactions of a neural network controller and a safety monitor in a simulated environment.\newline
A neural network was trained, tested and trained again several times with and without the safety monitor, to collect data about the emergence of these components.

\section{Measures of interest}

The goal of this work is to develop and to assess the feasability of an experimental method for the safety assessment of an autonomous vehicle. Due to the system being composed by two constituent systems: the Controller and the Checker, we think that a point of view based on the emergent behaviour resulting from the interaction of these systems can improve the quality of the assessment.\newline
The main aspects we are interested in are:

\begin{itemize}
	\item How the coverage of the system changes when the same monitor is applied to different stages of a network
	\item Changes in the safety gain provided by the same monitor when applied to two different networks
	\item What features of the monitor determine an improvement (or worsening) to the safety of the system
	\item Possible behaviours of the neural network having an impact on the monitor usefulness
\end{itemize}

At the system level, we want to minimize the probability of a safety-failure (e.g. a crash). As long as the network is trained properly, we expect that $P_{n_{i}}(failure) \geq P_{n_{i+j}}(failure)$ where $n_{i}$ represents a neural network trained for $i$ epochs. At the same time we want our safety monitor to provide at least the same level of fault tolerance if the network is improved.\newline
After the definition of $n$ scenarios in which the system must be tested, the safety of the controller was measured in the following way:

\begin{itemize}
	\item $TTF_{i,j}$ = Running time of a controller at epoch $i$ in the $j_{th}$ scenario 
	\item $MTTF_{Controller_{i}} = \frac{1}{n} \sum_{k = 0}^{n} TTF_{i,k}$
\end{itemize}

Since we are working in a simulated environment, the time to failure was computed using simulations steps, without loss of generality.\newline 

One of the main problem realated to assessing AVs' safety is the execution time. As said before, the $f_{i} = P(crash)_{at\_gen\_i}$ increases monotonically over time but, the increasing factor should be lower the more the network is trained. This property could result in very long simulations, but it also gives a useful hint for checking whether the system's safety is improving or not.\newline
Given a network $N$ in 2 generations $g_{1}, \quad g_{2}$ with $g_{2} > g_{1}$ meaning that generation $g_2$ had been trained more time, we can write:\newline\newline


Once trained neural networks become essentially black boxes and even a small variation on the training parameters could result in totally different behaviours during test phase, therefore it can not be assumed that the same software (the monitor) will provide the same level of safety.\newline
A safety monitor uses statistical models to model the environment around the car collecting data from the sensors. These data are used to predict the next state of the system and eventually put it in a safe state.\newline
A monitor's prediction can be one of the following:

\begin{itemize}
	\item True Negative
	\begin{itemize}
		\item Correct behaviour, no failure and no alarm raised
	\end{itemize}
	\item True Positive
	\begin{itemize}
		\item Failure in the Controller correctly detected and corrected by the Monitor
	\end{itemize}
	\item False Negative
	\begin{itemize}
		\item Failure of the Controller not detected by the Monitor
	\end{itemize}
	\item False Positive
	\begin{itemize}
		\item No failure but Monitor raises alarm
	\end{itemize}
\end{itemize}

Of course we desire the monitor's detections to be the most accurate possible. For this reason \textsl{Sensitivity}\footnote{True Positive rate: the proportion of safety measures applied by the monitor when actually needed} and False Negative rate were chosen as measures of interest.\newline
While in most of the cases a false positive will result in a state of degraded service, since a self-driving car is a safety-critical system performing in a dynamic environment, a false positive could put the system in an unsafe state (imagine performing an emergency brake for no reason on the highway), so False positive rate was taken into account as well.\newline

*****************************\newline 
TO REVIEW: 
\newline *************************\newline
This is why, for each failure, different data were recorded such as:
\begin{itemize}
	\item Whether or not the monitor raised an alarm
	\item The change in speed of the car after the alarm was raised
	\item If a collision with a vehicle V occured, the speed and the direction of V
\end{itemize}

In this way it's possible to measure more efficiently the overlapping between the set of safety hazards covered by the AI and the one covered by the Monitor,
(Other metrics: velocita` a cui andava la macchina), (direzione da cui veniva l'altro veicolo se incidente)
---> per capire le situazioni in cui sbaglia di piu`


\section{Experiments methodology}

In this section we propose and describe the methodology developed in this work.\newline

The study consists of several experiments in a simulated realistic environment, in which we observe how the coverage of the safety-monitor (i.e. the probability of raising an alert if there really is a safety-hazard) varies with respect to a neural network \textsl{in different stages of training}.\newline
The first step to perform the analysis is to define what are the metrics of interest.

	\begin{itemize}
		
		\item Come vengono effettuati gli esperimenti (scenari? durata fissa? ad oltranza? fino ad un fallimento? \dots)
		
	\end{itemize}

The efficacy of the constituent systems was studied separately at first. Since the goal of a vehicle is to go from point A to point B without crashing or hitting a pedestrian, the controller was tested at first, recording the the actions performed 