\chapter{Automotive - State of art}

\section{Safety in the Automotive}

-  Intro e standard

- Perche` le neural network sono un problema per la safety e perche` e` difficile validarla per questi sistemi| citazioni paperz (RAND study, koopmann, high-dependability systems\dots)



\subsection{Controller - Checker Problem}

======================= \newline
Grafico overlapping safety \newline
======================= \newline

As the network learns, we expect the area covered by the Primary to grow. With a relatively simple Monitor, in relatively simple scenarios, there will potentially be no overlapping between the hazard areas covered by the two. In this phase, the safety gain provided by the use of a (correctly implemented) safety checker will be remarkable, since the Primary is still learning to handle "easy" demands. As pointed in the previous sections, our main goal is to observe and study the variation of the dependability provided by the monitor when the network is trained to handle "hard" demands, since there are no guarantees on the Monitor's performance in the long period.\newline
As noted in \cite{striginiPopov} the probability of a failure for a \textsl{system} composed by a \textsl{Primary Component} and a \textsl{Safety-Monitor} on a random demand X is:

\begin{equation}
P_{fp} (1 - Coverage_{\sigma}) - covariance_{Q} (\theta (X), C_{\sigma} (\sigma , X))
\end{equation}

where:

\begin{itemize}
	\item $P_{fp} (1 - Coverage_{\sigma})$ is the probability of a failure in the Primary Component ($P_{fp}$) that is \textbf{not detected} by the Safety Monitor (the term $1 - Coverage_{\sigma}$ is exactly the probability of having a false negative/positive)
	\item $covariance_{Q} (\theta (X), C_{\sigma}, X)$ given a demand profile $Q = <x, y>$ (i.e. the pair <demand, output), measures the correlation between:
	\begin{itemize}
		\item[$\theta (X)$ -] The expected probability that the Primary will fail when processing demand $X$
		\item[$C_{\sigma} (\sigma, X)$ -] the term identifying the \textsl{Coverage Factor} of the \textbf{Monitor}, on the specific demand $X$
	\end{itemize}
\end{itemize}

This formula highlights the deep connection between the safety levels of the Controller and the Monitor, when it comes to the global safety of the system. It is clear from the equation that the probability of observing a failure in the system is also depending on the specific demand $X$, that's 
.\newline

***************\newline

TO REVIEW:\newline

***************\newline

The formula points the fact that to have the probability of observing a failure in the system depends not only on \textsl{all the possible demands} in the \textsl{demand space} but also on how the controller and the monitor react to them.\newline
This ideas, and the complete lack of literature or studies on the topic, put the basis for our study:

\begin{itemize}
	\item To prove the feasibility of a methodology to assess the safety level of systems composed by a \textsl{Primary Controller} and a \textsl{Safety Checker}
\end{itemize}


