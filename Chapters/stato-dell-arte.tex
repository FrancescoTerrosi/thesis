\chapter{Stato dell'arte}

\section{Self-driving cars}

Le macchine a guida autonoma rientrano nella categoria dei sistemi informatici cyber-fisici critici. Anche se gia` di per se` assimilabili nella categoria dei sistemi di sistemi (dal momento che vi sono piu` constituent systems che interagiscono fra loro), l'obiettivo della comunita` scientifica e` quello di riuscire ad ottenere un sistema di sistemi risultante dall'interconnessione di piu` veicoli autonomi: ognuno di questi sistemi avra` un obiettivo differente (i.e. diverse destinazioni da raggiungere) ma per ottenerlo e` assolutamente necessaria una cooperazione fra essi. Come diretta conseguenza, e` indispensabile che il sistema di controllo del veicolo sappia non solo obbedire al codice della strada ma anche essere in grado di riconoscere le situazioni di potenziale pericolo.\newline
Un veicolo autonomo dev'essere in gradio di "osservare" l'ambiente circostante, questo e` reso possibile grazie ai sensori installati su di essa.\newline\newline Tipicamente vengono utilizzati:
\begin{itemize}
	\item Telecamere
		\begin{itemize}
			\item[$\rightarrow$] Necessarie per catturare immagini dell'ambiente
		\end{itemize}
	\item Radar, Lidar, Sonar
		\begin{itemize}
			\item[$\rightarrow$] Utilizzati per creare una mappa dell'ambiente in cui naviga il veicolo e per percepire gli ostacoli
		\end{itemize}
	\item GPS, sensori inerziali, odometria
		\begin{itemize}
			\item[$\rightarrow$] Indispensabili per pianificare il percorso da seguire e conoscere la posizione del veicolo nell'ambiente operativo
		\end{itemize}
\end{itemize}

\newpage

L'architettura software di un veicolo e` composta da piu` moduli interagenti, dove l'errore di uno di questi potrebbe risultare in una minaccia per la safety del sistema.\newline\newline
Possiamo semplificare il modello architetturale come composto da tre moduli separati:


\begin{itemize}
	\item Environment Mapping
	\begin{itemize}
		\item[$-$] La componente che riceve i dati direttamente dai sensori e si occupa di filtrarli (e.g. per ridurne il rumore e scartare valori poco significativi) ed aggregarli per riconoscere il perimetro dell'area circostante ed eventuali ostacoli
	\end{itemize}
	\item Motion control
	\begin{itemize}
		\item[$-$] Questo modulo riceve in input i dati dopo che questi sono stati elaborati dal sistema di \textsl{data processing}. Dopo averli osservati, il \textsl{motion planner} decide in che direzione debba proseguire il veicolo; questo comando viene quindi inviato al \textsl{controller}, composto da un modulo di controllo longitudinale (accelerazione) e uno di controllo laterale (sterzo del veicolo), il quale va effettivamente a interagire con gli attuatori del sistema
	\end{itemize}
	\item System Supervisor
	\begin{itemize}
		\item[$-$] Un supervisore, o monitor, e` il sistema che si occupa di rilevare guasti o fallimenti sia hardware che software. Dal punto di vista hardware, i controlli effettuati sono principalmente sui guasti a componenti hardware e sugli output del \textsl{controller} (ad esempio che appartengano al dominio del sistema). Il monitor software invece si occupa di rilevare inconsistenze fra gli output dei due moduli precedenti. Questo e` di fondamentale importanza in quanto permette di controllare che l'output del \textsl{controller} non porti il sistema in una situazione di pericolo, o peggio: ad un fallimento catastrofico che coinvolgerebbe anche vite umane
	\end{itemize}
\end{itemize}


===================================================================

IMMAGINE ARCHITETTURA SISTEMA DI CONTROLLO
[SETTIMANA 1 - LEZIONE 3]

===================================================================
\newline\newline

Se classicamente venivano implementati in software modelli statistici noti per mappare l'ambiente (e.g. Kalman filter) e modelli fisici o di teoria del controllo per manovrare il veicolo (e.g. PID controller), grazie al progresso nel campo del \textsl{machine learning} avuto negli ultimi anni si e` iniziato a utilizzare reti neurali addestrate alla guida \cite{1} \cite{2}.\newline
Le reti neurali hanno dimostrato di sapere reagire meglio a situazioni sconosciute rispetto ai meccanismi classici, tuttavia richiedono tanti piu` dati quanto piu` e` complesso il compito da eseguire.\newline
Questi modelli computazionali inoltre soffrono del cosiddetto \textsl{black box problem}: e` molto difficile riuscire perche` la rete abbia associato l'output \textsl{y} all'input \textsl{x}. Nonostante siano stati proposti alcuni framework \cite{3} per aiutare a comprendere i meccanismi che regolano le decisioni di un'intelligenza artificiale sotto specifiche assunzioni, non si e` ancora trovata una soluzione universale al problema.\newline


\section{Safety nell'Automotive}

