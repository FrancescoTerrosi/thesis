\chapter{Automotive - State of art}

Self driving cars are one of the hottest topics of the decade. Artificial Intelligences specifically trained to drive with machine learning techniques demonstrated that it's possible for a computer to drive cars. However, a failure in these kind of systems may have very serious consequences that could result in people being injuried, or killed. At the same time, it is a problem to certify the ultra-high dependability requirements of these systems. In this chapter, today's problems regarding the safety issues related to self driving cars are reviewed, for that it was decided to conduct this study.

\section{Autonomous Cars as CPS}

\begin{itemize}
	\item Parlare di macchine autonome e come queste sono CPS
	\item Architettura alto livello: controller - monitor
\end{itemize}

\section{Safety in the Automotive}

The level of autonomy of a car ranges from 0 to 5. Level 0 means no autonomy: a human driver just drives the car, level 5 means that there's no need of human intervention at all and the car is not only capable of driving safely on the road, but it must be able to avoid catastrophic failures that may seriously harm (or kill) people.
The more autonomous the car is, the higher the dependability requirements are for it to be put on public roads.
It is well known that demonstrating a system's dependability is not an easy task for itself, it gets even harder with ultra-high dependability systems such as these are. In addition to the problem itself, demonstrating autonomous cars' dependability has two more problems to deal with: how to safely and effectively test the system and the need for neural networks to achieve the task.\newline 
Lots of studies demonstrated that it's unthinkable to just test cars on the roads. One of these, that we will refer as the RAND Study, answers the question of how many miles of driving would it take to demonstrate autonomous vehicles' reliability using classical statistical inference, saying that if autonomous cars fatality rate was 20\% lower than humans', it would take more than 500 years with \textsl{"a fleet of 100 autonomous vehicles being test-driven 24 hours a day, 365 days a year at an average speed of 25 miles per hour"}. It's just impossible to demonstrate cars' dependability by observing them driving and hope they won't fail. Not just for the quantitative results of the impossibility such as the RAND study, but also because failures and crash are (should be) very rare failure events, therefore it could take too much time for one of them to happen, or it may not happen at all during tests.\newline\newline

Validating these requirements is a hard task already. Things are made harder by the fact that these cars are driven by neural networks.\newline
In these years there is a huge interest in the \textsl{machine learning} sector, and this has made that a lot of progress was done in the research. It's also thanks to these progresses that autonomous cars now seem like something we can achieve, since these AIs gave surprising results with their skills and big names such as \textsl{Uber} and \textsl{Tesla} are putting more and more efforts in AI research.

If neural networks gave promising results on one hand, and they seem the only way to achieve goals such as autonomous cars, thanks to their ability to handle situations which they were not specifically programmed to handle, on the other hand the lack of official regulations and certifications of these kinds of software raised some concerns, especially after dramatic events such as the death of a woman in Arizona\cite{arizuber}, and consciousness is growing on the topic, asking for more regulations on companies developing advanced AIs\cite{elonmusk}





- Perche` le neural network sono un problema per la safety e perche` e` difficile validarla per questi sistemi| citazioni paperz (RAND study, koopmann, high-dependability systems\dots)


\section{neural network che guidano}


\section{Controller - Checker Problem}

==============================================================================================

Qui descriverei il nostro problema. Per adesso ho solo copincollato quello che avevo scritto in precedenza, userei questa sezione per giustificare cosa facciamo

=================================================================================================


As the network learns, we expect the area covered by the Primary to grow. With a relatively simple Monitor, in relatively simple scenarios, there will potentially be no overlapping between the hazard areas covered by the two. In this phase, the safety gain provided by the use of a (correctly implemented) safety checker will be remarkable, since the Primary is still learning to handle "easy" demands. As pointed in the previous sections, our main goal is to observe and study the variation of the dependability provided by the monitor when the network is trained to handle "hard" demands, since there are no guarantees on the Monitor's performance in the long period.\newline
As noted in \cite{striginiPopov} the probability of a failure for a \textsl{system} composed by a \textsl{Primary Component} and a \textsl{Safety-Monitor} on a random demand X is:

\begin{equation}
P_{fp} (1 - Coverage_{\sigma}) - covariance_{Q} (\theta (X), C_{\sigma} (\sigma , X))
\end{equation}

where:

\begin{itemize}
	\item $P_{fp} (1 - Coverage_{\sigma})$ is the probability of a failure in the Primary Component ($P_{fp}$) that is \textbf{not detected} by the Safety Monitor (the term $1 - Coverage_{\sigma}$ is exactly the probability of having a false negative/positive)
	\item $covariance_{Q} (\theta (X), C_{\sigma}, X)$ given a demand profile $Q = <x, y>$ (i.e. the pair <demand, output), measures the correlation between:
	\begin{itemize}
		\item[$\theta (X)$ -] The expected probability that the Primary will fail when processing demand $X$
		\item[$C_{\sigma} (\sigma, X)$ -] the term identifying the \textsl{Coverage Factor} of the \textbf{Monitor}, on the specific demand $X$
	\end{itemize}
\end{itemize}

This formula highlights the deep connection between the safety levels of the Controller and the Monitor, when it comes to the global safety of the system. It is clear from the equation that the probability of observing a failure in the system is also depending on the specific demand $X$, that's 
.\newline

***************\newline

TO REVIEW:\newline

***************\newline

The formula points the fact that to have the probability of observing a failure in the system depends not only on \textsl{all the possible demands} in the \textsl{demand space} but also on how the controller and the monitor react to them.\newline


