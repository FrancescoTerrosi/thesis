\chapter{Method Implementation And Results}

In this chapter the tools used, the infrastructure and method implementation and the results collected during the analysis are reviewed.

\section{Tools and software}

\subsection{Carla Simulator}

In order to have a realistic environment, with accurate physics simulation and data sensors, the open-source simulator CARLA\cite{carla}, developed by researchers at the University of Barcellona, was used. This simulator was developed with the purpose of offering an environment where AI agents can be trained to drive, with high control of the simulation parameters and the simulation of realistic sensor, which can be tuned to increase or decrease data quality, or to inject faults.\newline
CARLA is developed with a client-server architecture in mind. The \textsl{server} is basically a game, developed with \textsl{Unreal Engine 4} in C++. C++ performances are with no doubts essential to the functionality of the server: not only the environment must be simulated (inlcuding movements of pedestrians/vehicles, weather simulation\dots), but also all the data needed from the sensors attached to the system.\newline


======================================================================================

IMMAGINE CARLA

======================================================================================

CARLA is currently at version 0.9.7 and huge improvements are done at every release, gaining more attention from the experts for its realisticity. Unfortunately, when this study started, CARLA 0.9 was recently released and the tools needed for our work couldn't be found online. Thanks to the quantity of work done for the last \textit{stable} version of CARLA, 0.8.4 was used at first.\newline
Versions prior to 0.9 have some limitations on the control one has of the simulations parameters and on the data collectable from it. This doesn't impede our study, but of course limited in some way the informations on the environment and system. Some of these problems are still present in later versions of the simulator, but most of them were solved in the transition from 0.8 to 0.9.\newline\newline
One of the main problem found was with the coordinate systems. In CARLA v. 0.8 they use \textsl{left hand coordinates system}. Things could be easily solved by applying a transformation matrix. However, due to performance issues (a Python client should do the real-time processng of \textsl{loads} of data at each timestep, resulting in considerable slowdowns), it was decided to stick with the developers' decision and convert the data during analysis phase.



\begin{itemize}
	\item CARLA
	\item Nervana Systems - coach (Intel)
	\item Reti neurali su git
	\item Monitor
\end{itemize}


\section{Method Implementation}

Dedicare una sezione alle decisioni prese?

\section{Results}

\begin{itemize}
	
	\item Interazione rete-monitor
	\item Safety Monitor Implementation - obstacle detection
	\item Come vengono raccolti i dati
	\item Come vengono preprocessati
	
\end{itemize}

