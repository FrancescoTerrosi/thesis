\paragraph{Abstract}\mbox{}\\*\\*

I veicoli autonomi sono sistemi cyber-fisici critici, complessi sotto molteplici aspetti: dalla tecnologia necessaria all'acqusizione di dati esterni come radar, lidar, GPS\dots all'implementazione del software che si occupera` della guida vera e propria. Con l'enorme progresso avuto nel campo del machine learning nell'ultimo ventennio, la prospettiva di macchine capaci di guidare senza alcuna interazione con l'uomo e` sempre piu` vicina.\newline
Il sistema di controllo del veicolo puo` essere visto in maniera semplificata come formato da una rete neurale, che determina l'azione da eseguire (quanto accelerare/decelerare o l'angolo di sterzata) sulla base dei dati ricevuti dai sensori. Dal momento che nelle predizioni effettuate da una rete vi e` insito un errore (i.e. risulta impossibile avere un'accuratezza del $100\%$ sui risultati prodotti) e` di fondamentale importanza avere un \textsl{safety monitor}, il cui compito e` quello di controllare e sanificare gli output dell'intelligenza artificiale.\newline In questo lavoro abbiamo studiato come varia il rapporto fra un safety-monitor relativamente semplice e una rete neurale addestrata per la guida autonoma, andando a definire dei semplici requisiti di safety e osservando come un continuo training della rete neurale vada a impattare (o meno) sull'utilita` del monitor in questione.\newline
Per poter svolgere questo lavoro sono stati utilizzati molti software open-source: grazie al simulatore CARLA e` stato possibile avere una rappresentazione realistica delle leggi della strada e della fisica dei veicoli. Le reti neurali che sono poi state prese in considerazione sono state addestrate con algoritmi di \textsl{reinforcement learning} e \textsl{imitation learning}, considerati fra i piu` promettenti in questo campo. CARLA inoltre permette di avere una simulazione realistica dei sensori utilizzati sui veicoli autonomi: questo ha permesso di costruire un semplice (ma efficace) safety monitor che effettua dei controlli di sicurezza sulla base dei dati ricevuti dal lidar e sulla velocita` e direzione del veicolo autonomo; sono stati infine condotti gli esperimenti per studiare l'interazione fra questi due componenti.